\documentclass[12pt]{article}
\usepackage{amsmath, amssymb, graphicx}
\usepackage[a4paper, margin=1in]{geometry}

\title{\textbf{Data Structures Worksheet: Array-Based Stack \& Circular Queue}}
\author{}
\date{}

\begin{document}

\maketitle

\section*{Students Names: \underline{\hspace{8cm}}}

\section*{Part 1: Array-Based Stack}

\subsection*{Conceptual Questions}
\begin{enumerate}
    \item Explain the Last-In-First-Out (LIFO) property of a stack. Why is it useful in computing?
    
    \vspace{3cm}
    
    \item Describe the key operations of a stack (\texttt{push}, \texttt{pop}, \texttt{peek}). What are their time complexities?
    
    \vspace{3cm}
    
    \item What happens when you push an element onto a full stack? How would you handle this situation?
    
    \vspace{3cm}
    
    \item Given a stack with an initial state of $[ \_, \_, \_, \_, \_ ]$ (capacity = 5), perform the following operations and show the stack state after each step:
    \begin{itemize}
        \item push(5)
        \item push(10)
        \item push(15)
        \item pop()
        \item push(20)
        \item push(25)
        \item push(30) (What happens here?)
    \end{itemize}
    
    \vspace{3cm}
    
    \item Stacks can be used for function call management in programming languages. Explain why this is the case and how the stack helps in function execution.
    
    \vspace{3cm}
    
    \item Consider a postfix expression evaluation. How would a stack help in computing $5\ 3\ +\ 8\ *\ 2\ -$?
    
    \vspace{3cm}
\end{enumerate}

\newpage

\section*{Part 2: Array-Based Circular Queue}

\subsection*{Conceptual Questions}
\begin{enumerate}
    \item Explain how a circular queue differs from a standard queue.
    
    \vspace{3cm}
    
    \item Why do we use modular arithmetic in implementing a circular queue?
    
    \vspace{3cm}
    
    \item What is the condition to check if the circular queue is full? How do we distinguish between an empty and a full queue?
    
    \vspace{3cm}
    
    \item Consider a circular queue of max size 5. Given the operations below, trace the state of the queue, including the front and rear pointers:
    \begin{itemize}
        \item enqueue(1)
        \item enqueue(2)
        \item enqueue(3)
        \item enqueue(4)
        \item enqueue(5)
        \item dequeue()
        \item dequeue()
        \item enqueue(6)
        \item enqueue(7)
        \item dequeue()
        \item enqueue(8)
        \item enqueue(9) (What happens here?)
    \end{itemize}
    
    \vspace{3cm}
    
    \item Suppose a circular queue of size 6 is initially empty. Show the changes in the queue state and the front/rear indices after performing these operations:
    \begin{itemize}
        \item enqueue(10), enqueue(20), enqueue(30)
        \item dequeue()
        \item enqueue(40), enqueue(50), enqueue(60), enqueue(70)
        \item dequeue(), dequeue()
        \item enqueue(80), enqueue(90)
        \item What happens when we attempt another enqueue(100)? Explain.
    \end{itemize}
    
    \vspace{3cm}
    
    \item In real-world applications, where would a circular queue be more efficient than a standard queue? Provide an example.
    
    \vspace{3cm}
\end{enumerate}

\section*{Submission Instructions}
\begin{itemize}
    \item Complete all conceptual questions with a partner.
    \item Fill in the stack and circular questions and draw tracing tables.
    \item Submit a document with your answers at the end of class.
    \item Ensure your names are on the document.
\end{itemize}

\textbf{Due Date:} \underline{\hspace{6cm}}

\end{document}

