\documentclass{article}
\usepackage{amsmath, amssymb}
\usepackage{listings}
\usepackage{xcolor}

% Define code listing style
\lstdefinestyle{mystyle}{
    numbers=left,
    numbersep=5pt,
    numberstyle=\tiny\color{gray},
    basicstyle=\ttfamily\footnotesize,
    keywordstyle=\color{blue},
    commentstyle=\color{green},
    stringstyle=\color{red},
    frame=single,
    breaklines=true,
    tabsize=4
}
\lstset{style=mystyle}

\begin{document}

\title{CS 152 Midterm Review Guide}
\author{}
\date{}
\maketitle

\section{Conceptual Questions}

\textbf{1. Understanding Object-Oriented Principles}
\begin{itemize}
    \item What is abstraction in object-oriented programming? What is the smallest and largest unit of abstraction? How is abstraction implemented in Python?
    \item What is encapsulation? Why is it important? How is encapsulation implemented in Python?
    \item You are building a video game like Grand Theft Auto using object-oriented programming. One of your first tasks is to design the Vehicle representation. Describe your design using an object-oriented approach.
\end{itemize}

\textbf{2. Complexity Analysis}
\begin{itemize}
    \item List the complexity analysis values starting from the most efficient to the least efficient.
    \item Why is algorithm efficiency analyzed using Big-O notation?
    \item Describe the performance complexity of:
    \begin{itemize}
        \item Iterating through a two-dimensional grid of unknown size.
        \item Iterating through an array.
        \item Sequential search of an array.
        \item Accessing an item in an array.
    \end{itemize}
\end{itemize}


\textbf{3. Data Structure Design Scenario}
\begin{itemize}
    \item You are designing a sequential data structure that is fast yet flexible for storing ordered elements. You have implemented fixed-size arrays and a bag so far, and you know about stacks and queues.
    \item Describe the internal structure of your data structure.
    \item List the functions the ADT should support and their time complexities.
\end{itemize}

\section{Code Question}

\textbf{Task:} Implement a static method to merge two Array objects as described below.

\begin{lstlisting}[caption={Merge two Array objects}, label={lst:merge_arrays}]
@staticmethod
def merge(array1: "Array", array2: "Array") -> "Array":
    """
    Merges two Array objects in a single order fashion. 
    If either array is longer, merge the items in the back of the resulting array.

    Example:
        >>> print(array1)
        [5, 7, 17, 13, 11]
        >>> print(array2)
        [12, 10, 2, 4, 6]
        >>> new_array = Array.merge(array1, array2)
        >>> print(new_array)
        [5, 12, 7, 10, 17, 2, 13, 4, 11, 6]

    Args:
        array1 (Array): The first array to merge.
        array2 (Array): The second array to merge.

    Returns:
        Array: A new array containing the merged elements from array1 and array2.

    Raises:
        TypeError: If either array1 or array2 are not Array objects.
    """
    pass  # Implementation required
\end{lstlisting}



\end{document}

