
\documentclass[12pt]{article}
\usepackage{amsmath, amssymb}
\usepackage{fancyhdr}
\usepackage{listings}
\usepackage{xcolor}
\usepackage[margin=1in]{geometry}
\usepackage{titlesec}
\usepackage{enumitem}

\titleformat{\section}{\large\bfseries}{\thesection}{1em}{}
\titleformat{\subsection}{\normalsize\bfseries}{\thesubsection}{1em}{}
\setlist{nosep}

\pagestyle{fancy}
\fancyhf{}
\rhead{CS 152: Data Structures}
\lhead{Recursive Functions Worksheet}
\rfoot{\thepage}

\lstset{
  basicstyle=\ttfamily\small,
  keywordstyle=\color{blue},
  commentstyle=\color{gray},
  stringstyle=\color{red},
  showstringspaces=false,
  breaklines=true,
  frame=single,
  tabsize=4,
  language=Python
}

\begin{document}

\begin{center}
    \Large \textbf{CS 152 Data Structures}\\
    \large Recursive Functions Worksheet
\end{center}

\vspace{1em}

\textbf{Name:} \rule{10cm}{0.4pt}

\vspace{1em}

\section*{Background}

Recursive functions are functions that call themselves to solve a problem. A recursive function typically has:
\begin{itemize}
    \item A \textbf{base case} that stops the recursion.
    \item A \textbf{recursive case} that reduces the problem and calls the function again.
\end{itemize}

Understanding recursive functions is a key skill for solving problems like traversing trees, generating permutations, and many others.

\section*{Part 1: Predicting Output}

\textbf{Problem 1.} What does the following function return when called with \texttt{mystery(4)}?

\begin{lstlisting}
def mystery(n):
    if n == 0:
        return 1
    else:
        return n * mystery(n - 1)
\end{lstlisting}

\vspace{4em}

\textbf{Problem 2.} What does the following function return when called with \texttt{countdown(3)}?

\begin{lstlisting}
def countdown(n):
    if n == 0:
        return "Liftoff!"
    else:
        return str(n) + " " + countdown(n - 1)
\end{lstlisting}

\vspace{4em}

\section*{Part 2: Convert to Iteration}

Rewrite the \texttt{mystery} function from Problem 1 using a \texttt{while} loop instead of recursion.

\textbf{Your Solution Below:}

\vspace{8em}

\section*{Part 3: Recursive Function Practice}

Write a recursive function called \texttt{next\_prime\_after\_double(n)} that takes an integer \texttt{n}, doubles it, and returns the next prime number that is greater than the doubled value.

\textbf{Steps:}
\begin{itemize}
    \item Write a helper function \texttt{is\_prime(num)} to check if a number is prime.
    \item Double the input \texttt{n}.
    \item Recursively find the next prime number greater than \texttt{2 * n}.
\end{itemize}

\vspace{1em}

\textbf{Algorithm for \texttt{is\_prime(num)}}:
\begin{itemize}
    \item Any number less than 2 is not prime.
    \item A number is prime if it has no divisors other than 1 and itself.
    \item To check this efficiently, try dividing the number by all integers from 2 up to the square root of the number.
    \item If any of these divisions results in a remainder of 0, the number is not prime.
\end{itemize}

\vspace{10em}

\textbf{Optional:} Try writing an iterative version of the same function.

\vspace{4em}

\section*{Part 4: Reflection}

In your own words, reflect on the usefulness of recursive functions. When are they helpful, and when might they be harder to work with than loops?

\textbf{Prompt:} What did you learn about recursion from this worksheet? When might you prefer a recursive solution over an iterative one, and what are some possible downsides?

\vspace{10em}

\textit{End of Worksheet}

\end{document}
