\documentclass[pdf, aspectratio=169, 12pt]{beamer}
\usepackage[]{hyperref, graphicx, siunitx, lmodern, tikz, booktabs, physics}
\usepackage[mode=buildnew]{standalone}
\usepackage{pdfpc-commands}

\usetheme[sols]{Python}

\graphicspath{ {Images/} }

\sisetup{per-mode=symbol}
\usetikzlibrary{calc, patterns, decorations.markings, decorations.pathmorphing, shapes}

%Preamble
\title{For once again we shall loop}
\author{Jed Rembold}
\date{February 10, 2020}

\begin{document}

\begin{frame}{Announcements}
	\begin{itemize}
		\item Homework
			\begin{itemize}
				\item I'm still working on getting all of HW2 scored, but hopefully by the end of today!
				\item Homework 3 has been posted! Should be good to do Problems 1 and 2 after today.
			\end{itemize}
		\item Polling: \url{rembold-class.ddns.net}
	\end{itemize}
\end{frame}

\begin{frame}{Review Question}
	Which of the following questions would be a suitable candidate for using some sort of exhaustive enumeration type algorithm?
	\begin{poll}
	\item Finding all multiples of 3 that are also divisible by 5
	\item Counting the number of vowels in a sentence
	\item Determining the precise amount of time to fire a rockets thrusters to reach a certain speed
	\item Calculating the cubic root of 50
	\end{poll}
	\exsol{Counting the number of vowels in a sentence}
\end{frame}

\begin{frame}{Example 2: Substring exists?}
	\begin{itemize}
		\item Countable solution set?
			\begin{itemize}
				\item Just need to check each piece of the bigger string.
				\item It is made up of a countable number of parts, so we just need to check a countable number
			\end{itemize}
		\item Terminating Condition?
			\begin{itemize}
				\item Assume \pyi{i=0} initially and increments by one
				\item Taking the big string to be \pyi{b_str} and the substring to be \pyi{l_str}, keep going until
					\begin{center}
						\pyi{i + len(l_str) > len(b_str)}
					\end{center}
			\end{itemize}
	\end{itemize}
\end{frame}

\begin{frame}{Example 3: Money Money}
	\begin{itemize}
		\item Countable solution set?
			\begin{itemize}
				\item We have a limited amount of coins to check
				\item Just need to check all possible combinations
			\end{itemize}
		\item Terminating Condition(s)?
			\begin{itemize}
				\item Probably want one for each loop here
				\item Assume \pyi{num_nic}, \pyi{num_dime}, \pyi{num_pen} all start at zero and increment by one
				\item Stop when
					\begin{center}
						\pyi{num_nic > max_nic}\\
						\pyi{num_dime > max_dime}\\
						\pyi{num_pen > max_pen}
					\end{center}
			\end{itemize}
	\end{itemize}
\end{frame}

\begin{frame}[fragile]{Iterating over sequences}
	\begin{itemize}
		\item It comes up often that we iterate over some sequence or range
			\begin{pythoncode}
				i = 0
				while i < 10:
					print(i)
					i = i + 1
			\end{pythoncode}
		\item Python provides us with a simplified type of loop to do these sorts of iterations
		\item Called a \alert{for} loop
			\begin{itemize}
				\item Can think of it as saying: ``\alert{For} each thing in this stuff, run this code.''
			\end{itemize}
	\end{itemize}
\end{frame}

\begin{frame}[fragile]{Anatomy of a For Loop}
	\begin{itemize}[<+->]
		\item For loops have a simple syntax:
			\begin{pythoncode}
				for <variable> in <sequence>:
					<loop code>
			\end{pythoncode}
		\item \pyi{<loop code>} is no different from what we have already been doing inside loops
		\item \pyi{<variable>} is the variable name that will take on every value in the sequence over the course of the loop
		\item \pyi{<sequence>} is a non-scalar object who individual pieces will be looped over
	\end{itemize}
\end{frame}

\begin{frame}[fragile]{Simple String Example}
	\vspace{5mm}
	\begin{itemize}
		\item We already know strings are non-scalar objects, so they can be looped over with for loops!
			\begin{pythoncode}
				dessert = "Pumpkin Pie"
				for letter in dessert:
					print(letter)
			\end{pythoncode}
		\item This is functionally identical to:
			\begin{pythoncode}
				dessert = "Pumpkin Pie"
				i = 0
				while i < len(dessert):
					print(dessert[i])
					i = i + 1
			\end{pythoncode}
	\end{itemize}
\end{frame}

\begin{frame}{Far Ranging Sequences}
	\begin{itemize}
		\item How can this be useful with numbers?
		\item Python's \pyi{range()} function
			\begin{itemize}
				\item Can pass in the same options as slicing
					\begin{itemize}
						\item Start (defaults to 0)
						\item Stop
						\item Step (defaults to 1)
					\end{itemize}
				\item Like slicing, the \pyi{Stop} element will not be included
				\item Unlike slicing though, use a comma (\pyi{,}) to separate the options!!
				\item Generates values ``as needed'', so very memory efficient
			\end{itemize}
	\end{itemize}
\end{frame}

\begin{frame}[fragile]{For Ranging Examples}
	\begin{itemize}
		\item Providing just a stop
			\begin{pythoncode}
				for n in range(5):
					print(n)
			\end{pythoncode}
		\item Providing start and stop
			\begin{pythoncode}
				for n in range(1,11):
					print(n)
			\end{pythoncode}
		\item Providing start, stop and step
			\begin{pythoncode}
				for n in range(10,0,-1):
					print(n)
			\end{pythoncode}
	\end{itemize}
\end{frame}

\begin{frame}[fragile]{Understanding Check}
	Which of the below blocks of code would print something different than the others?
	\begin{poll}
		\small
		\begin{columns}[t]
			\column{0.45\textwidth}
	\item 
		\begin{pythoncode}
			for n in range(10):
				if n % 2 == 0:
					print(n)
		\end{pythoncode}

	\item
		\begin{pythoncode}
			for i in range(0,10,2):
				if i:
					print(i)
		\end{pythoncode}
			
			\column{0.45\textwidth}
	\item
		\begin{pythoncode}[showlines=true]
			for j in range(0,20,4):
				print(j // 2)
				  
		\end{pythoncode}

	\item
		\begin{pythoncode}
			for k in range(0,10):
				if not k % 2:
					print(k)
		\end{pythoncode}
			
		\end{columns}

	\end{poll}
	\exsol{B}
\end{frame}

%\begin{frame}[fragile]{Pythagorean Integers: For Loop Version}
	%\vspace{5mm}
	%\begin{itemize}
		%\item Let's revisit our problem of finding all positive integers $a$ and $b$ such that
			%\[a^2 + b^2 = 250\]
	%\end{itemize}
	%\vspace{-5mm}
	%\begin{columns}[t]
		%\column{0.5\textwidth}
		%\begin{block}{While Loops}
			%\scriptsize
			%\begin{pythoncode}[basicstyle=\color{FG}\ttfamily]
				%a = 0
				%while 250 - a > 0:
					%b = 0
					%while 250 - b > 0:
						%if a**2 + b**2 == 250:
							%print('-----')
							%print('A=',a)
							%print('B=',b)
						%b = b + 1
					%a = a + 1
			%\end{pythoncode}
		%\end{block}
		%\column{0.5\textwidth}
		%\begin{block}{For Loops}
			%\scriptsize
			%\begin{pythoncode}[basicstyle=\color{FG}\ttfamily]
				%for a in range(250):
					%for b in range(250):
						%if a**2 + b**2 == 250:
							%print('-----')
							%print('A=',a)
							%print('B=',b)
			%\end{pythoncode}
		%\end{block}
	%\end{columns}
%\end{frame}

%\begin{frame}{Making the Choice}
	%\begin{columns}[t]
		%\column{0.5\textwidth}
			%\textcolor{Teal}{While Loops:}
				%\begin{itemize}
					%\item Very general and flexible
					%\item Can check and terminate on any condition you can imagine
					%\item You are responsible for initializing and updating needed variables
					%\item Need to be careful of infinite loops
					%\item Can mimic behavior of any \pyi{for} loop
				%\end{itemize}
		%\column{0.5\textwidth}
			%\textcolor{Teal}{For Loops:}
			%\begin{itemize}
				%\item Only iterate over sequences
				%\item Variable initialization and updating is handled by the sequence
				%\item Impossible to get an infinite loop
				%\item Simpler syntax in general
				%\item Can not mimic every \pyi{while} loop
			%\end{itemize}
	%\end{columns}
%\end{frame}

%\begin{frame}{More Involved Example}
	%Let \pyi{s} be a string that contains a sequence of decimal numbers separated by commas. For example, \pyi{s = 1.23,4.67,8.37}. Write a program that prints the sum of the numbers in \pyi{s}.
%\end{frame}
















\end{document}

