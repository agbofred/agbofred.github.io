\documentclass[pdf, aspectratio=169, 12pt]{beamer}
\usepackage[]{hyperref, graphicx, siunitx, lmodern, tikz, booktabs, physics}
\usepackage[mode=buildnew]{standalone}
\usepackage{pdfpc-commands}
\usepackage{pgfplots}
\pgfplotsset{compat=1.16}

\usetheme[sols]{Python}

\graphicspath{ {Images/} }

\sisetup{per-mode=symbol}
\usetikzlibrary{calc, patterns, decorations.markings, decorations.pathmorphing, shapes}


% NOTE TO SELF!!!!!!!!!!
% Moving this lecture 1 day ahead of Newton-Raphson to better fit homework timelines

%Preamble
\title{Function over Form}
\author{Jed Rembold}
\date{February 17, 2020}

\begin{document}

\begin{frame}{Announcements}
	\begin{itemize}
		\item Homework
			\begin{itemize}
				\item I'm still grading HW3, sorry. Hoping to get scores send out by tonight
				\item HW4 has been posted. You should be able to start looking at Problem 3 after today.
			\end{itemize}
		\item Polling: \url{rembold-class.ddns.net}
	\end{itemize}
\end{frame}

\begin{frame}{Review Question}
	How would the fraction $\tfrac{9}{16}$ be represented in binary floating point?
	\begin{poll}
	\item (1001, -100)
	\item (111, -101)
	\item (1111, -111)
	\item (1100, 101)
	\end{poll}
	\exsol{(1001, -100)}
\end{frame}

%\begin{frame}{Review Question}
	%In which of the below situations would using the Newton-Raphson approach to find a solution be potentially useful?
	%\begin{poll}
		%\item Finding an integer solution $a$ and $b$ for $a^2 + b^2 = 450$. 
		%\item Finding two prime numbers that sum to 100
		%\item Finding the square root of a number
		%\item Finding the number of ways in which 5 chips can be drawn from a bag containing chips of 7 different colors
	%\end{poll}
%\end{frame}

\begin{frame}{The Story So Far}
	\begin{itemize}
		\item We've already looked at
			\begin{itemize}
				\item numbers
				\item variable assignments
				\item input and output
				\item comparisons
				\item loops
			\end{itemize}
		\item This enough to be Turing complete!
		\item What more do we want?
			\begin{itemize}
				\item Improved usability!
					\begin{itemize}
						\item Ability to reuse code
						\item Ability to generalize code
					\end{itemize}
			\end{itemize}
	\end{itemize}
\end{frame}

\begin{frame}{Function Objectives}
	\vspace{5mm}
	\begin{itemize}
		\item<1-> Already familiar with a small host of built-in functions:
			\begin{itemize}
				\item \pyi{print}
				\item \pyi{abs}
				\item \pyi{len}
				\item etc
			\end{itemize}
		\item<2-> You don't care what code happens when you run those, you care that they perform a particular task for you
		\item<3-> You pass them some input (or inputs) and expect the expression to return some output (or outputs)
			\begin{itemize}
				\item Looks like: \texttt{<output> = function(<input>)}
				\item \pyi{5 == abs(-5)}
				\item \pyi{11 == len('long string')}
			\end{itemize}
		\item<4-> Would be very useful to be able to define our own functions
	\end{itemize}
\end{frame}

\begin{frame}[fragile]{Defining Functions}
	\begin{itemize}[<+->]
		\item We can define a new function like:
			{\small
				\begin{pythoncode}[moredelim={[is][\color{Blue}]{@}{@}}]
					def @<name of function>@ (@<list of formal parameters>@):
						@<body of function>@
				\end{pythoncode}
			}
		\item Syntax: Starts with \pyi{def}, then function name, then parenthesis, then colon
		\item Pieces:
			\begin{itemize}
				\item name of function: mandatory -- what you want the function to be called
				\item list of formal parameters: optional -- potential parameters or ``inputs'' you want to pass to the function
				\item body of function: mandatory -- whatever you want the function to do. May or may not include a \pyi{return} statement
			\end{itemize}
	\end{itemize}
\end{frame}

\begin{frame}[fragile]{Varied Uses}
	\vspace{5mm}
	\begin{itemize}[<+->]
		\item As a way to group code
			\begin{pythoncode}
				def print_stuff():
					print('Stuff 1')
					print('Stuff 2')
			\end{pythoncode}
		\item As a way to provide input to a group of code
			\begin{pythoncode}
				def print_mult_of_A(A):
					print(A, 2*A, 3*A, 4*A, 5*A)
			\end{pythoncode}
		\item As a machine that takes some input and provides some output for later use
			\begin{pythoncode}
				def calc_year_savings(init_amt):
					for i in range(12):
						init_amt += init_amt*0.03/12
					return init_amt
			\end{pythoncode}
	\end{itemize}
\end{frame}

\begin{frame}{Functional Vocab}
	\begin{description}
		\item[Formal Parameters:] The variable names used in the defining of the function which will be assigned to actual literals upon function call
		\item[Actual Parameters or Arguments:] The numbers or other literals passed to the function which are assigned the formal parameter variable names
		\item[Function Definition:] Where the function is actually defined, in terms of formal parameters
		\item[Function Call (or Invocation):] Where the function is used in the code, with accompanying arguments passed to the function. 
	\end{description}
\end{frame}

\begin{frame}[fragile]{The Flow}
	\begin{columns}
		\column{0.6\textwidth}
		{\footnotesize
			\begin{pythoncode}[numbers=left]
			|def year_savings(init_amt, int_rate):|
				for i in range(12):
					init_amt *= (1+int_rate/12)
				return init_amt

			|amount = 100
			rate = 0
			while rate < 1:
				print("Savings:", |
					|year_savings(amount, rate))|
				|rate += 0.05|

			print('All done!')
		\end{pythoncode}
		}
		\column{0.4\textwidth}
		\begin{itemize}
			\item<1,5> Function definition with formal parameters (1)
			\item<2,5> Normal code (6-9)
			\item<3,5> Function call with current \pyi{amount} and \pyi{rate} bound to formal parameters (9-10)
				\begin{itemize}
					\item Iterate across 12 months calculating compound interest (2-3)
					\item Return final value (4)
				\end{itemize}
			\item<4,5> Resume main code block and increase rate (11)
		\end{itemize}
	\end{columns}
\end{frame}

\begin{frame}{Common Mistakes}
	\begin{itemize}[<+->]
		\item Make sure you have parentheses in your function definition, even if you are not supplying any parameters!
		\item That colon and indenting your function code block are still mandatory, just like with \pyi{if} statements or loops
		\item You do not need a \pyi{return} statement, the function will automatically return \pyi{None} if it reaches the end of the function code without encountering one
		\item \pyi{return} statements will break out of the function code as soon as they are called, rejoining the main code flow.
			\begin{itemize}
				\item You can't return something midway through a function and then return more things later. It all has to be at once.
				\item You can use this in certain situations as a clever \pyi{break} statement, getting you out of a loop earlier than you otherwise would
			\end{itemize}
	\end{itemize}
\end{frame}

\begin{frame}{Example: Summing Primes}
	\begin{example}
		Let's look at the problem where we'd like to find all the positive pairs of prime numbers that sum to 100. Here we'll define a function called \pyi{is_prime} to help us out and greatly simplify our code.
	\end{example}
\end{frame}

%\begin{frame}[fragile]{The Word is Key}
	%\vspace{5mm}
	%\begin{itemize}[<+->]
		%\item So far we have looked at a positional way to assign arguments to formal parameters
			%\begin{itemize}
				%\item The first argument to the first parameter, the second to the second, etc
					%\begin{center}
						%\begin{pythoncode}
							%def func(first, second, third):
								%print(first, second, third)

							%func(1,2,3)
							%func(2,6,4)
						%\end{pythoncode}
					%\end{center}
			%\end{itemize}
		%\item Can also write explicitly with keyword arguments!
			%\begin{itemize}
				%\item If you do so, the position no longer matters
					%\begin{center}
						%\begin{pythoncode}
							%func(third=4, first=2, second=6)
						%\end{pythoncode}
					%\end{center}
			%\end{itemize}
		%\item \textcolor{Red}{All keyword arguments must come after any positional arguments!}
	%\end{itemize}
%\end{frame}

%\begin{frame}[fragile]{Default Slide Title}
	%\begin{itemize}[<+->]
		%\item Can also specify default values for a formal parameter
			%\begin{pythoncode}
				%def func(name='Jed', age=34)
					%print('My name is', name, 'and I am', age)
			%\end{pythoncode}
		%\item You then don't need to always provide that actual parameter
		%\item If setting any parameters out of order though, you \alert{must} indicate them through keywords.
			%\begin{pythoncode}
				%func()
				%func('Bob',25)
				%func('Larry')
				%func(age=68)
			%\end{pythoncode}
	%\end{itemize}
%\end{frame}





\end{document}

