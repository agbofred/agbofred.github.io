\documentclass[pdf, aspectratio=169, 12pt]{beamer}
\usepackage[]{hyperref, graphicx, siunitx, lmodern, tikz, booktabs, physics}
\usepackage[mode=buildnew]{standalone}
\usepackage{pdfpc-commands}

\usetheme{Python}

\graphicspath{ {Images/} }

\sisetup{per-mode=symbol}
\usetikzlibrary{calc, patterns, decorations.markings, decorations.pathmorphing, shapes}

%Preamble
\title{Coding (can be) Exhausting}
\author{Jed Rembold}
\date{February 7, 2020}

\begin{document}

\begin{frame}{Announcements}
	\begin{itemize}
		\item Homework
			\begin{itemize}
				\item HW2 due tonight!
				\item Remember to change the ``Assignment status:'' to DONE!
				\item I'm aiming to get HW3 posted tomorrow or early Sunday
			\end{itemize}
		\item Polling: \url{rembold-class.ddns.net}
	\end{itemize}
\end{frame}

%\begin{frame}{Homework Comments}
	%\begin{itemize}
		%\item All HW1 has been graded. Comments and scores uploaded to GitHub.
		%\item Most common issues:
			%\begin{itemize}
				%\item Not working in the provided file itself, so formatting errors when copying and pasting back into template.
				%\item Not following instructions about what to print (usually not that big of a deal when I check myself, but can mask other errors)
				%\item Having ``holes'' in blocks of conditional statements.
				%\item Comments are optional\ldots until they cost you points.
			%\end{itemize}
		%\item That said, most people did quite well
	%\end{itemize}
%\end{frame}


%\begin{frame}[fragile]{Review Question}
	%Suppose you have the string: \pyi{x = "consternation"} and you'd like to just extract and print the word \pyi{"nation"}. Which expression below will \alert{not} give you the string \pyi{"nation"}?

	%\begin{poll}
	%\item \pyi{x[7:len(x)]}
	%\item \pyi{x[7:]}
	%\item \pyi{x[-6:len(x)]}
	%\item \pyi{x[-6:-1]}
	%\end{poll}
%\end{frame}


%\begin{frame}{And now the Dicing}
	%\begin{itemize}
		%\item Realize slices are a form of shorthand for looping through a string and pulling out the value between two indices
		%\item But what if you don't want \emph{every} character between the two endpoints?
			%\begin{itemize}
				%\item Only want every other character? Or every 10th character?
			%\end{itemize}
		%\item Can achieve by adding one more term to our slices:
			%\begin{center}
				%\pyi{x[<start>:<stop>:<step>]}
			%\end{center}
			%\begin{itemize}
				%\item A \pyi{<step>} of 2 would take every other character
				%\item Default is a \pyi{<step>} of 1
				%\item A \emph{negative} step will go backwards from stop to start!
			%\end{itemize}
	%\end{itemize}
%\end{frame}

%\begin{frame}[fragile]{Can't change a string's colors}
	%\begin{itemize}
		%\item Strings are what we call \alert{immutable}: they can not be modified in place
		%\item You can ``look'' at different parts of the string, but you can not ``change'' those parts
			%\begin{itemize}
				%\item Phased differently, you can't reassign the various pieces of a string
			%\end{itemize}
			
			%\begin{pythoncode}
				%s = "Cats!"

				%s[0] = "R"			# This will error!!
			%\end{pythoncode}
		%\item You can of course still reassign \pyi{s} to some new string object
			%\begin{pythoncode}
				%s = "R" + s[1:]
			%\end{pythoncode}
	%\end{itemize}
%\end{frame}

\begin{frame}[fragile]{Review Question}
	\vspace{5mm}
	Suppose I included all the necessary code to import arcade, create a window of \pyi{WIDTH} and \pyi{HEIGHT}, run it, etc. Inside the start and finish render portions, I include the below code. What image would be drawn to the window?
	\footnotesize
	\begin{pythoncode}
		i = 0
		while i < WIDTH:
			A = i / WIDTH * 360
			arcade.draw_arc_outline(i, i, 20, 20, arcade.color.BLACK, 
											A, A+90, 15)
			i = i + 30
	\end{pythoncode}
	\begin{center}
		\begin{tikzpicture}
			\foreach \x/\L in {0/A, 3/B, 6/C, 9/D}{
				\node(i) at (\x,0) {\includegraphics[width=2.75cm]{Images/ArcadeImg\L.png}};
				\node[below, black, font=\Large\bf] at (i.north) {\L};
			}
		\end{tikzpicture}
	\end{center}
\end{frame}

\begin{frame}{Moving On}
	\begin{itemize}
		\item For the next 2 weeks we'll focus on common algorithms or approaches to solving problems
		\item Most code (with one exception) will mostly just use the pieces we've already learned
		\item Algorithms will mostly focus on trying to answer more numeric types of problems
	\end{itemize}
\end{frame}

\begin{frame}{Guess and Check the Python Way}
	\begin{itemize}[<+->]
		\item Suppose we were tasked with solving the (very simple) problem
			\[(x + 12)^2 = 36\]
			where we wanted to find any integer solutions of \pyi{x}.
		\item Instead of doing math, we could ``guess'' values of \pyi{x} and then check to see if we got the correct answer
		\item This would be a terrible way for \alert{you} to solve the problem, but computers are king of solving tedious problems
		\item Looking for solutions in this way is called \alert{exhaustive enumeration}
	\end{itemize}
\end{frame}

\begin{frame}{Why guess?}
	\begin{itemize}
		\item No doubt, exhaustive enumeration is usually not the most efficient way to find an answer, so why do it?
			\begin{itemize}
				\item Algorithm is easy to read and understand
				\item Tends to be simple to implement
				\item Can easily give whole sets of solutions that meet conditions
				\item Modern computers are \textbf{fast}
					\begin{itemize}
						\item No sense spending hours coming up with a super efficient algorithm if it only saves you 5 milliseconds on runtime
					\end{itemize}
					
			\end{itemize}
			
	\end{itemize}
	
\end{frame}


\begin{frame}{Exhaustive Tips}
	\begin{itemize}
		\item Important pieces to consider before using exhaustive enumeration:
			\begin{itemize}
				\item You need to be able to (in theory) list out every single possible value that might be solution
					\begin{itemize}
						\item Mathematically \emph{countable}
						\item Integers are a good example, as you could list them all out (even a huge number of them)
						\item Real numbers or floats would \emph{not} work, as they are uncountable
					\end{itemize}
				\item You need to be able to come up with a loop conditional which assures that your loop will end, even if a solution is not found.
					
			\end{itemize}
	\end{itemize}
\end{frame}

%\begin{frame}{Decrementing Functions}
	%\vspace{5mm}
	%\begin{itemize}
		%\item Just a more formal way of looking at the condition we place on our \pyi{while} loop
		%\item Requirements:
			%\begin{itemize}
				%\item Maps variables to an integer
				%\item Non-negative when the loop starts
				%\item Loop terminates when value is $\leq 0$
				%\item Value is decreased each iteration through the loop
			%\end{itemize}
		%\pause
		%\item In our example, our decrementing function would have been
			%\[36 - x\]
		%\item Can be useful for more complex conditions to check the decrementing function (print it to the screen) to ensure that it is indeed decrementing
	%\end{itemize}
%\end{frame}

%\begin{frame}{Understanding Check}
	%Suppose I wanted to solve the problem:
	%\[x^3 = 216\]
	%where \pyi{x} would start at 0 and increment by one each iteration. What decrementing function would be appropriate?
	%\begin{poll}
	%\item \pyi{x ** 3 + 216}
	%\item \pyi{216 + x ** 3}
	%\item \pyi{x ** 3 - 216}
	%\item \pyi{216 - x ** 3}
	%\end{poll}
%\end{frame}

\begin{frame}{Understanding Check}
	Which of the following questions would \emph{not} be a suitable candidate for using some sort of exhaustive enumeration type algorithm?
	\begin{poll}
	\item Determining if a 3 digit number is prime
	\item Checking if a particular substring exists within a larger string
	\item Given 3 nickels, 2 dimes, and 8 pennies, finding all possible combinations totalling 25 cents
	\item Finding all values of $a$ and $b$ where $a^2 + b^2 = 250$
	\end{poll}
	\exsol{Finding all values of $a$ and $b$ where $a^2 + b^2 = 250$}
\end{frame}

\begin{frame}{Example 1: 3 digit prime?}
	\begin{itemize}
		\item Countable solution set?
			\begin{itemize}
				\item For any number ($n$), we would just need to check all the values smaller than that number to see if they divide the larger number equally
				\item Set of values to check from 2 $\rightarrow$ $n-1$, so countable
			\end{itemize}
		\item Terminating condition?
			\begin{itemize}
				\item Start at \pyi{x=2}
				\item Stop when \pyi{x == n} (yes, there are more optimal ways)
			\end{itemize}
	\end{itemize}
\end{frame}

\begin{frame}{Example 2: Substring exists?}
	\begin{itemize}
		\item Countable solution set?
			\begin{itemize}
				\item Just need to check each piece of the bigger string.
				\item It is made up of a countable number of parts, so we just need to check a countable number
			\end{itemize}
		\item Terminating Condition?
			\begin{itemize}
				\item Assume \pyi{i=0} initially and increments by one
				\item Taking the big string to be \pyi{b_str} and the substring to be \pyi{l_str}, keep going until
					\begin{center}
						\pyi{i + len(l_str) > len(b_str)}
					\end{center}
			\end{itemize}
	\end{itemize}
\end{frame}

\begin{frame}{Example 3: Money Money}
	\begin{itemize}
		\item Countable solution set?
			\begin{itemize}
				\item We have a limited amount of coins to check
				\item Just need to check all possible combinations
			\end{itemize}
		\item Terminating Condition(s)?
			\begin{itemize}
				\item Probably want one for each loop here
				\item Assume \pyi{num_nic}, \pyi{num_dime}, \pyi{num_pen} all start at zero and increment by one
				\item Stop when
					\begin{center}
						\pyi{num_nic > max_nic}\\
						\pyi{num_dime > max_dime}\\
						\pyi{num_pen > max_pen}
					\end{center}
			\end{itemize}
	\end{itemize}
\end{frame}

%\begin{frame}[fragile]{Exhaustive Letters}
	%\begin{itemize}
		%\item It is easy to check a wide range of numbers: we just add one each time
		%\item How to check a range of letters?
			%\begin{itemize}
				%\item Check a range of numbers and use indexing to associate those numbers with letters
			%\end{itemize}
			%\pause
			%\begin{pythoncode}
				%letters = 'abcdefghijklmnopqrstuvwxyz'
				%i = 0
				%while letters[i] != secret_letter:
					%i = i + 1
				%print('The secret letter was: ', letters[i])
			%\end{pythoncode}
			
	%\end{itemize}
	
%\end{frame}









\end{document}

