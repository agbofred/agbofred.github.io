\documentclass[pdf, aspectratio=169, 12pt]{beamer}
\usepackage[]{hyperref, graphicx, siunitx, lmodern, tikz, booktabs, physics}
\usepackage[mode=buildnew]{standalone}
\usepackage{pdfpc-commands}

\usetheme{Python}

\graphicspath{ {Images/} }

\sisetup{per-mode=symbol}
\usetikzlibrary{calc, patterns, decorations.markings, decorations.pathmorphing, shapes}

%Preamble
\title{While Slicing and Dicing}
\author{Jed Rembold}
\date{February 3, 2020}

\begin{document}

\begin{frame}{Announcements}
	\begin{itemize}
		\item Homework
			\begin{itemize}
				\item Homework 2 has been posted!
				\item I may well still be working on getting through grading all your HW1. Hopefully by the end of today all feedback will be posted
				\item If you are still working on HW1, try to get it in soon! Late days/hours are burning away!
			\end{itemize}
		\item Remember that tutors are available Monday and Thursday nights, from 7-10pm
		\item Polling: \url{rembold-class.ddns.net}
	\end{itemize}
\end{frame}


\begin{frame}[fragile]{Review Question}
	\vspace{4mm}
	What will the output of the below code be upon completion?
	\begin{pythoncode}
		x = 7
		y = 0
		while x >= 0:
			y = y + x
			x = x - 2
		print(y)
	\end{pythoncode}
	\begin{poll}
		\item 15
		\item 16
		\item 28
		\item Code will not complete (infinite loop)
	\end{poll}
	\exsol{16}
\end{frame}

\begin{frame}[fragile]{Lost Woods Examples}
	\begin{columns}
		\column{0.5\textwidth}
		\begin{pythoncode}[style=output,]
			************************
			************************
						:(
			************************
			************************
			Go (L)eft or (R)ight?
		\end{pythoncode}
		
		\column{0.5\textwidth}
		\begin{pythoncode}[style=output,]
			**********     **********
			**********     **********
						 :)
			**********     **********
			**********     **********
			Go (N)orth, (S)outh, 
			(E)ast, or (W)est?
		\end{pythoncode}
	\end{columns}
\end{frame}

\begin{frame}[fragile]{Nesting Loops}
	\begin{columns}
		\column{0.4\textwidth}
		\begin{itemize}
			\item You are allowed to nest loops, and in many cases might want to
			\item Realize that one loop must finish \alert{before} the next advances one iteration!
				\begin{itemize}
					\item At which point the inner loop would run in its entirety again
				\end{itemize}
		\end{itemize}
		
		\column{0.6\textwidth}
		\begin{pythoncode}
			week = 1
			day = 1

			while week <= 52:
				while day <= 3:
					if day == 1:
						print('Monday')
					elif day == 2:
						print('Wednesday')
					elif day == 3:
						print('Friday')

					day = day + 1
				week = week + 1
		\end{pythoncode}
	\end{columns}
\end{frame}

\begin{frame}[fragile]{Taking a break}
	\vspace{5mm}
	\begin{itemize}
		\item Sometimes it can be useful to exit a loop without checking the initial test
			\begin{itemize}
				\item Maybe you have multiple conditions that could stop the loop
				\item Maybe you have a special condition that when certain requirements are met the loop ends
			\end{itemize}
		\item The \pyi{break} statement will force an immediate exit from whatever loop it happens to be in
	\end{itemize}
	\begin{pythoncode}
		x = 1
		while True:
			print(x)
			x = x + 1
			if x > 10:
				break
	\end{pythoncode}
\end{frame}

\begin{frame}[fragile]{Understanding Check}
	What would be the \emph{last} number printed by the below code?
	\begin{columns}
		\column{0.5\textwidth}
		\begin{pythoncode}
			x = 1
			while x < 10:
				y = 10
				while y > x:
					if y % 3 == 0:
						break
					print(x*y)
					y = y - 2
				x = x + 1
		\end{pythoncode}
		
		\column{0.5\textwidth}
		\begin{poll}
		\item 90
		\item 60
		\item 56
		\item 25
		\end{poll}
	\end{columns}
	\exsol{90}
\end{frame}

\begin{frame}{Dissecting Strings}
	\vspace{5mm}
	\begin{itemize}
		\item Recall that strings are a non-scalar object
			\begin{itemize}
				\item Made up of smaller pieces, \alert{characters} in the case of strings
			\end{itemize}
		\item Each character has an ``address'' within the string where it lives, called its \alert{index}.
	\end{itemize}
	\begin{center}
		\begin{tikzpicture}
			\node[font=\Huge\ttfamily, SynPurple, anchor=south west] at (0,0) {"Spaghetti"};
			\foreach[count=\t] \x in {.8,1.26,...,4.5}{
				\node<2->[font=\small] at (\x, -0.5) {\t};
			}
			\draw<4->[Red, very thick] (0,-.5) -- (5.5,-0.5);
			\foreach[count=\t from 0] \x in {.8,1.26,...,4.5}{
				\node<4->[font=\small, Green] at (\x, -1.0) {\t};
			}
			\foreach[count=\t from -9] \x in {.8,1.26,...,4.5}{
				\node<5->[font=\small, Teal] at (\x, -1.5) {\t};
			}
		\end{tikzpicture}
	\end{center}
			\begin{alertblock}<3->{Start at 0!}
				Counting (and indexing) in Python always starts with 0!! This is easy to forget!
			\end{alertblock}
\end{frame}

\begin{frame}{Indexing Notation}
	\begin{itemize}
		\item To get the character at a certain address or index, use \emph{square brackets}
	\begin{center}
		\begin{tikzpicture}
			\node[font=\Huge\ttfamily, SynPurple, anchor=south west] at (0,0) {x = "Spaghetti"};
			\foreach[count=\t from 0] \x in {.8,1.26,...,4.5}{
				\node[font=\small, Green] at (\x+1.8, -0.5) {\t};
			}
		\end{tikzpicture}
	\end{center}
	\begin{itemize}
		\item \pyi{x[0] = "S"}
		\item \pyi{x[4] = "h"}
		\item \pyi{x[-1] = "i"}
		\item \pyi{x[1] = "p"}
	\end{itemize}
	\end{itemize}
\end{frame}

\begin{frame}{The Slicing}
	\begin{itemize}
		\item Frequently, you want more than a single character
		\item Would be useful to give a start and stop point to extract a ``sub''-string
			\begin{itemize}
				\item Possible through \alert{slicing}!
				\item Notation is: \pyi{var[<start>:<stop>]}
				\item Could be read as: ``Get me all the characters from the \pyi{<start>} up to the \pyi{<stop>} \emph{but not including the stop}''
			\begin{center}
				\begin{tikzpicture}
					\node[font=\Huge\ttfamily, SynPurple, anchor=south west] at (0,0) {x = "Spaghetti"};
					\foreach[count=\t from 0] \x in {.8,1.26,...,4.5}{
						\node[font=\small, Green] at (\x+1.8, -0.5) {\t};
					}
				\end{tikzpicture}
			\end{center}
			\item \pyi{x[0:3] = "Spa"}
			\item \pyi{x[4:] = "hetti"}
			\end{itemize}
	\end{itemize}
\end{frame}

\begin{frame}{and the Dicing}
	\begin{itemize}
		\item Slices are a form of shorthand for looping through a string and pulling out the value between two indices
		\item But what if you don't want \emph{every} character between the two endpoints?
			\begin{itemize}
				\item Only want every other character? Or every 10th character?
			\end{itemize}
		\item Can achieve by adding one more term to our slices:
			\begin{center}
				\pyi{x[<start>:<stop>:<step>]}
			\end{center}
			\begin{itemize}
				\item A \pyi{<step>} of 2 would take every other character
				\item Default is a \pyi{<step>} of 1
				\item A \emph{negative} step will go backwards from stop to start!
			\end{itemize}
	\end{itemize}
\end{frame}




\end{document}

