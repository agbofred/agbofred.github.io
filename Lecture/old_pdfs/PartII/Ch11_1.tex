\documentclass[pdf, aspectratio=169, 12pt]{beamer}
\usepackage[]{hyperref, graphicx, siunitx, lmodern, tikz, booktabs, physics, multicol}
\usepackage[mode=buildnew]{standalone}
\usepackage{pdfpc-commands}
\usepackage{pgfplots}
\pgfplotsset{compat=1.16}

\usetheme[sols]{Python}

\graphicspath{ {Images/} }

\sisetup{per-mode=symbol}
\usetikzlibrary{calc, patterns, decorations.markings, decorations.pathmorphing, shapes, fit}

%Preamble
\title{All about the plot}
\author{Jed Rembold}
\date{April 10, 2020}

\begin{document}

\begin{frame}{Announcements}
	\begin{itemize}
		\item Homework
			\begin{itemize}
				\item Homework 8 and 9 due tonight!
				\item Only 1 more homework!
			\end{itemize}
		\item Starting in on visualization today (Ch 11)
		\item Will be getting information to you next week concerning projects
		\item Polling: \nolinkurl{rembold-class.ddns.net}
	\end{itemize}
\end{frame}

\begin{frame}[fragile]{Review Question}
	\begin{columns}
		\column{0.5\textwidth}
		\small
		\begin{pythoncode}
			class MyClass:
				varA = 3
				varB = True

				def __init__(self):
					self.v = self.varA
					if self.varB:
						self.varA += 1

			A = MyClass()
			B = MyClass()
			MyClass.varB = False
			C = MyClass()
			print(MyClass.varA)
		\end{pythoncode}
		\column{0.5\textwidth}
		Suppose the code to the left was written and executed. What would be the output of the printed statement?
		\begin{poll}
		\item 3
		\item 5
		\item 6
		\item None of the above
		\end{poll}
		\exsol{3}
	\end{columns}
\end{frame}

\begin{frame}{Visualization}
	\begin{itemize}
		\item Interfaces mostly text so far:
			\begin{itemize}
				\item Written to screen
				\item Written to a file
				\item Input from screen or from file
			\end{itemize}
		\item Want to extend how we can communicate results or interface with our programs
			\begin{itemize}
				\item Data representation and plotting
				\item More complex graphical output/input
			\end{itemize}
	\end{itemize}
\end{frame}

\begin{frame}{Reminder: Using Modules}
	\begin{itemize}
		\item We already know how to import and write our own modules
		\item Many topics going forward will focus heavily on extending Python through the use of a common or core module
			\begin{itemize}
				\item Modules are usually written in classes!
				\item Create new objects with special attributes and properties
				\item Interact with those objects through specific methods
				\item \alert{Read the documentation} for a module or a method to learn what is available and how it should be used
			\end{itemize}
	\end{itemize}
\end{frame}

\begin{frame}[fragile]{Matplotlib}
	\begin{itemize}
		\item \emph{The} fundamental plotting and data visualization package for Python
			\begin{itemize}
				\item Old! Released 16 years ago!
			\end{itemize}
		\item Has two main ways to interface with the objects:
			\begin{itemize}
				\item A state-based interface based on MATLAB
				\item \alert<2>{An object-oriented interface}
			\end{itemize}
		\item Even using the OO interface, the state-based library has some useful bits, so that is still what we will import:
			\begin{pythoncode}
				import matplotlib.pyplot as plt
			\end{pythoncode}
	\end{itemize}
\end{frame}

\begin{frame}[fragile]{The Objects}
	\begin{itemize}
		\item Two primary objects you'll use in Matplotlib:
			\begin{itemize}
				\item A figure
					\begin{itemize}
						\item The overall image that is created
					\end{itemize}
					\begin{pythoncode}
						fig = plt.figure()
					\end{pythoncode}
				\item An axis
					\begin{itemize}
						\item A coordinate system into which data can be displayed
						\item Is inserted into a figure
						\item You can have multiple axes in a single figure if desired
					\end{itemize}
					\begin{pythoncode}
						ax = fig.add_axes()
						# or, more commonly
						ax = fig.add_subplot(111)
					\end{pythoncode}
			\end{itemize}
		%\item Can always consult documentation with \pyi{help(func)} or \pyi{func?}, but the online documentation is more convenient sometimes:
			%\begin{itemize}
				%\item \link{https://matplotlib.org/3.1.1/api/axes_api.html}{Axes Object Documentation}
			%\end{itemize}
	\end{itemize}
\end{frame}

\begin{frame}{Actually Plotting}
	\begin{itemize}
		\item You can add a plot to a desired axes instance
			\begin{itemize}
				\item Use the \pyi{.plot()} method
			\end{itemize}
		\item Plots are 2D
			\begin{itemize}
				\item Need a sequence of y coordinates
				\item Give a sequence of x coordinates or indices will be default
				\item x coordinates are given first if they exist
			\end{itemize}
		\item Style of the plot can be controlled with special format string as second/third parameter.
		\item Additional properties of a plot can be controlled with extra keyword parameters.
	\end{itemize}
\end{frame}

\begin{frame}[fragile]{Plot Customization}
	\begin{itemize}
		\item Can determine how a plot generally looks with a format string:
			\begin{itemize}
				\item Takes the form of
					\begin{pythoncode}
						'[marker][line][color]'
					\end{pythoncode}
					\begin{itemize}
						\item Common markers:   \pyi{. o ^ * s D}
						\item Common lines:   \pyi{- -- :}
						\item Common colors:   \pyi{b g r c k}
					\end{itemize}
					
			\end{itemize}
		\item Can also add keyword arguments:
			\begin{itemize}
				\item Change color: \pyi{color = 'green'}
				\item Charge marker size: \pyi{markersize = 20}
				\item Charge opacity: \pyi{alpha = 0.4}
				\item Add label for legend: \pyi{label = 'Best plot'}
			\end{itemize}
			
			
	\end{itemize}
	
\end{frame}


\begin{frame}{Extra Figure and Axes Features}
	\begin{itemize}
		\item Clear communication is important with visualization
		\item Should always include meaningful and descriptive axes labels and figure titles.%
			\begin{itemize}
				\item Axes labels controlled with \pyi{.set_xlabel()} and \pyi{.set_ylabel()}
				\item Axes title controlled with \pyi{.set_title}
				\item Figure title controlled with \pyi{.suptitle}
			\end{itemize}
		\item Adjust tick spacing or labels:
			\begin{itemize}
				\item Where ticks appear: \pyi{.set_xticks()} or \pyi{.set_yticks()}
				\item What labels they have: \pyi{.set_xticklabels()} or \pyi{.set_yticklabels()}
			\end{itemize}
	\end{itemize}
\end{frame}




\end{document}

